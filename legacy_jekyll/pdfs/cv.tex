\documentclass[11pt,]{article}
\usepackage[sc, osf]{mathpazo}
\usepackage{amssymb,amsmath}
\usepackage{ifxetex,ifluatex}
\usepackage{fixltx2e} % provides \textsubscript
\ifnum 0\ifxetex 1\fi\ifluatex 1\fi=0 % if pdftex
  \usepackage[T1]{fontenc}
  \usepackage[utf8]{inputenc}
\else % if luatex or xelatex
  \ifxetex
    \usepackage{mathspec}
  \else
    \usepackage{fontspec}
  \fi
  \defaultfontfeatures{Ligatures=TeX,Scale=MatchLowercase}
\fi
% use upquote if available, for straight quotes in verbatim environments
\IfFileExists{upquote.sty}{\usepackage{upquote}}{}
% use microtype if available
\IfFileExists{microtype.sty}{%
\usepackage{microtype}
\UseMicrotypeSet[protrusion]{basicmath} % disable protrusion for tt fonts
}{}
\usepackage[margin=1in]{geometry}


\usepackage{longtable,booktabs}


\setlength{\emergencystretch}{3em}  % prevent overfull lines
\providecommand{\tightlist}{%
  \setlength{\itemsep}{0pt}\setlength{\parskip}{0pt}}
\setcounter{secnumdepth}{0}
% Redefines (sub)paragraphs to behave more like sections
\ifx\paragraph\undefined\else
\let\oldparagraph\paragraph
\renewcommand{\paragraph}[1]{\oldparagraph{#1}\mbox{}}
\fi
\ifx\subparagraph\undefined\else
\let\oldsubparagraph\subparagraph
\renewcommand{\subparagraph}[1]{\oldsubparagraph{#1}\mbox{}}
\fi

% Now begins the stuff that I added.
% ----------------------------------

% Custom section fonts
\usepackage{sectsty}
\sectionfont{\rmfamily\mdseries\large\bf}
\subsectionfont{\rmfamily\mdseries\normalsize\itshape}


% Make lists without bullets
\renewenvironment{itemize}{
  \begin{list}{}{
    \setlength{\leftmargin}{1.5em}
  }
}{
  \end{list}
}


% Make parskips rather than indent with lists.
\usepackage{parskip}
\usepackage{titlesec}
\titlespacing\section{0pt}{12pt plus 4pt minus 2pt}{4pt plus 2pt minus 2pt}
\titlespacing\subsection{0pt}{12pt plus 4pt minus 2pt}{4pt plus 2pt minus 2pt}

% Use fontawesome. Note: you'll need TeXLive 2015. Update.
\usepackage{fontawesome}

% Fancyhdr, as I tend to do with these personal documents.
\usepackage{fancyhdr,lastpage}
\pagestyle{fancy}
\renewcommand{\headrulewidth}{0.0pt}
\renewcommand{\footrulewidth}{0.0pt}
\lhead{}
\chead{}
\rhead{}
\lfoot{
\cfoot{\scriptsize  Garrett Honke - Curriculum Vitae }}
\rfoot{\scriptsize \thepage/{\hypersetup{linkcolor=black}\pageref{LastPage}}}

% Always load hyperref last.
\usepackage{hyperref}
\PassOptionsToPackage{usenames,dvipsnames}{color} % color is loaded by hyperref

\hypersetup{unicode=true,
            pdftitle={Garrett Honke:  Curriculum Vitae (Curriculum Vitae)},
            pdfauthor={Garrett Honke},
            colorlinks=true,
            linkcolor=blue,
            citecolor=Blue,
            urlcolor=blue,
            breaklinks=true, bookmarks=true}
\urlstyle{same}  % don't use monospace font for urls

\begin{document}


\centerline{\huge \bf Garrett Honke}

\vspace{2 mm}

\hrule

\vspace{2 mm}

\moveleft.5\hoffset\centerline{Computational Neuroscientist}
\moveleft.5\hoffset\centerline{Mountain View, California}
\moveleft.5\hoffset\centerline{ \faEnvelopeO \hspace{1 mm} \href{mailto:}{\tt \href{mailto:garretthonke@gmail.com}{\nolinkurl{garretthonke@gmail.com}}} \hspace{1 mm}     \faGlobe \hspace{1 mm} \href{http://ghonk.github.io}{\tt ghonk.github.io}    | \emph{Updated:} \today}

\vspace{2 mm}

\hrule


\section{Education}\label{education}

\emph{Binghamton University (SUNY)}

PhD Cognitive and Brain Sciences \hfill 2012 - 2017

MSc Cognitive and Brain Sciences \hfill 2012 - 2015

\bigskip

\emph{University of Texas at Austin}

BA Psychology \hfill 2004 - 2008

\section{Positions}\label{positions}

Staff Research Scientist, Google \hfill 2024 - Present

Staff Research Scientist, Google X \hfill 2022 - 2024

Senior Research Scientist, Google X \hfill 2021 - 2022

Research Scientist, Google X \hfill 2019 - 2021

Research Scientist, New Knowledge \hfill 2018 - 2019

Postdoctoral Research Associate, Brain and Machine Laboratory
\hfill 2017 - 2018 \linebreak Co-appointment at the Watson School of
Engineering and Applied Science and \newline the Department of
Psychology: Cognitive and Brain Sciences \newline  Director: Dr.~Sarah
Laszlo; Binghamton University (SUNY)

Graduate Student, Brain and Machine Laboratory \hfill 2016 - 2017
\linebreak Director: Dr.~Sarah Laszlo; Binghamton University (SUNY)

Graduate Student, Learning and Representation in Cognition Laboratory
\hfill 2012 - 2017 \linebreak Director: Dr.~Kenneth J. Kurtz; Binghamton
University (SUNY)

Adult Lab Coordinator, Cognition and Language Laboratory \hfill 2010 -
2012 \linebreak Director: Dr.~Dedre Gentner; Northwestern University

Research Assistant, Similarity and Cognition Laboratory \hfill 2005 -
2007 \linebreak Director: Dr.~Arthur B. Markman; University of Texas at
Austin

\section{Refereed Publications and
Presentations}\label{refereed-publications-and-presentations}

Kirchenbauer, J., Honke, G., Somepalli, G., Geiping, J., Ippolito, D.,
Lee, K., Goldstein, T., \& Andre, D. (2024). LMD3: Language Model Data
Density Dependence. CoLM 2024

Patterson, J. D., Snoddy, S., Honke, G., Premo, J., Silliman, D. C.,
Cavagnetto, A. R., \& Kurtz, K. J. (2024). Improving concept learning in
education via category construction. \emph{Journal of Educational
Psychology, 116}(8), 1455-1478.

Edwards, C., Lai, T. M., Ros, K., Honke, G., Cho, K., and Ji, H. (2022)
Translation between Molecules and Natural Language. \emph{Proceedings of
the 2022 Conference on Empirical Methods in Natural Language
Processing.} EMNLP 2022.

Honke, G., Higgins, I., Thigpen, N., Miskovic, V., Link, K., Duan, S.,
Gupta, P., Klawohn, J., \& Hajcak, G. (2021). Representation learning
for improved interpretability and classification accuracy of clinical
factors from EEG. arXiv:2010.15274. ICLR 2021.

Cakmak, A. S., Thigpen, N., Honke, G., Alday, E. P., Rad, A, B.,,
Adaimi, R., Chang, C. J., Li, Q., Gupta, P., Neylan, T., McLean, S. A.,
\& Clifford, G. D. (2020). Using Convolutional Variational Autoencoders
to Predict Post-Trauma Health Outcomes from Actigraphy Data. NeurIPS
2020 ML4MH workshop, accepted as a spotlight talk.

Honke, G., Kurtz, K. J., \& Laszlo, S. (2020). Similarity Judgments
Predict N400 Amplitude Differences between Taxonomic Category Members
and Thematic Associates. \emph{Neuropsychologia, 141}, 107388.

Kurtz, K. J., \& Honke, G. (2020). Sorting out the problem of inert
knowledge: Category construction to promote spontaneous transfer.
\emph{Journal of Experimental Psychology: Learning, Memory, and
Cognition.46}(5), 803--821.

Dhamani, N., Azunre, P., Corcoran, C., Honke, G., Gleason, J. L.,
Kramer, S., \& Morgan, J. (2019). Using Deep Networks and Transfer
Learning to Address Disinformation. ICML 2019 AI for Social Good
Workshop.

Honke, G. \& Kurtz, K. J. (2019). Similarity is as Similarity Does? A
Critical Inquiry into the Effect of Thematic Association on Similarity.
\emph{Cognition, 186}, 115-138.

Gentner, D., Simms, N., Kurtz, K. J., Honke, G., Snoddy, S., Forbus, K.
D., Richland, L. E., Matlen, B. J., Lyons, E. M., \& Klostermann, E.
(2018). Relational Categories: Why they're Important and How they're
Learned. In C. Kalish, M. Rau, T. Rogers, \& J. Zhu (Ed.),
\emph{Proceedings of the 40\textsuperscript{th} annual conference of the
Cognitive Science Society} (pp.~27-28). Austin, TX: Cognitive Science
Society.

Premo, J., Cavagnetto, A. R., Honke, G., \& Kurtz, K. J. (2018).
Categories in Conflict: Combating the application of an intuitive
conception of inheritance with category construction. \emph{Journal of
Research in Science Teaching, 0}, 1-21.

Azunre, P., Corcoran, C., Sullivan, D., Honke, G., Ruppel, R., Verma,
S., \& Morgan, J. (2018). Abstractive Tabular Dataset Summarization via
Knowledge Base Semantic Embeddings. arXiv:1804.01503 {[}cs.AI{]}. ICML
2018 AutoML workshop.

Honke, G. R., Conaway, N. B., \& Kurtz, K. J. (2016). Switch it up:
Learning categories via feature switching. In A. Papafragou, D. Grodner,
D. Mirman, \& J. Trueswell (Eds.), \emph{Proceedings of the
38\textsuperscript{th} annual conference of the Cognitive Science
Society} (pp.~2693-2698). Austin, TX: Cognitive Science Society.

Gentner, D., Levine, S. C., Ping, R., Isaia, A., Dhillon, S., Bradley,
C., \& Honke, G. (2016). Rapid learning in a children's museum via
analogical comparison. \emph{Cognitive Science, 40}(1), 224-240.

Honke, G., Cavagnetto, A. R., Kurtz, K. J., Patterson, J. D., Conaway,
N. B., Tao, Y., \& Marr, J. C. (2015). Promoting Transfer and Mastery of
Evolution Concepts with Category Construction. Paper presented at the
American Educational Research Association annual meeting, Chicago, IL.

Gentner, D., Goldwater, M. B., Levine, S. C., Ping, R. M., Isiah, A.,
Honke, G., \& Bradley, C. (2015). Spatial language and spatial
comparison combine to support children's learning. \emph{Cognitive
Processing, 16}, S38-S38.

\section{Patents}\label{patents}

\href{https://scholar.google.com/citations?view_op=view_citation\&hl=en\&user=WPewiKcAAAAJ\&sortby=pubdate\&citation_for_view=WPewiKcAAAAJ:WbkHhVStYXYC}{Processing
time-domain and frequency-domain representations of eeg data} +
\href{https://scholar.google.com/citations?view_op=view_citation\&hl=en\&user=WPewiKcAAAAJ\&sortby=pubdate\&citation_for_view=WPewiKcAAAAJ:Tiz5es2fbqcC}{Resampling
eeg trial data} +
\href{https://scholar.google.com/citations?view_op=view_citation\&hl=en\&user=WPewiKcAAAAJ\&sortby=pubdate\&citation_for_view=WPewiKcAAAAJ:u9iWguZQMMsC}{EEG
signal representations using auto-encoders} +
\href{https://scholar.google.com/citations?view_op=view_citation\&hl=en\&user=WPewiKcAAAAJ\&sortby=pubdate\&citation_for_view=WPewiKcAAAAJ:XiSMed-E-HIC}{Attention
encoding stack in aggregation of data} +
\href{https://scholar.google.com/citations?view_op=view_citation\&hl=en\&user=WPewiKcAAAAJ\&sortby=pubdate\&citation_for_view=WPewiKcAAAAJ:p2g8aNsByqUC}{Attention
encoding stack in EEG trial aggregation} +
\href{https://scholar.google.com/citations?view_op=view_citation\&hl=en\&user=WPewiKcAAAAJ\&sortby=pubdate\&citation_for_view=WPewiKcAAAAJ:OU6Ihb5iCvQC}{Processing
eeg data with twin neural networks} +
\href{https://scholar.google.com/citations?view_op=view_citation\&hl=en\&user=WPewiKcAAAAJ\&sortby=pubdate\&citation_for_view=WPewiKcAAAAJ:uWQEDVKXjbEC}{Processing
time-frequency representations of EEG data using neural networks} +
\href{https://scholar.google.com/citations?view_op=view_citation\&hl=en\&user=WPewiKcAAAAJ\&sortby=pubdate\&citation_for_view=WPewiKcAAAAJ:SP6oXDckpogC}{De-noising
task-specific EEG signals using neural networks} +
\href{https://scholar.google.com/citations?view_op=view_citation\&hl=en\&user=WPewiKcAAAAJ\&sortby=pubdate\&citation_for_view=WPewiKcAAAAJ:UxriW0iASnsC}{Latent
Factor Structuring of Psychopathology}

\section{Invited Talks, Non-refereed Posters and
Presentations}\label{invited-talks-non-refereed-posters-and-presentations}

Corcoran, C., DiResta, R., Morar, D., Honke, G., Dhamani, N., Sullivan,
D., Gleason, J., Azunre, P., Kramer, S., Ruppel, B. (2019).
Disinformation: Detect to Disrupt. Comparative Approaches to
Disinformation workshop hosted by the Berkman Klien Center for Internet
and Society, Harvard University.

How I spent my summer vacation: Latin American Coldplay Bots take on
MTV's Hottest. A primer on analytics for the detection and investigation
of coordinated online disinformation campaigns. Texas Analytics Summit
2018, hosted by the Center for Research Analytics at the McCombs Schoool
of Business, University of Texas at Austin.

Kurtz, K. J., Cavagnetto, A. R., Honke, G., Conaway, N. B., Patterson,
J. D., Marr, J. C. \& Tao, Y. (2014). Optimizing the category
construction task to promote learning and transfer of knowledge in
classroom instruction. In P. Bello, M. Guarini, M. McShane, \& B.
Scassellati (Eds.), \emph{Proceedings of the 36\textsuperscript{th}
Annual Conference of the Cognitive Science Society.} Austin, TX:
Cognitive Science Society.

Kurtz, K. J., \& Honke, G. (2013). Self-generated analogies promote
spontaneous transfer. Poster presented at the 54\textsuperscript{th}
annual meeting of the Psychonomic Society, Toronto, ON.

Honke, G., Gentner, D., Forbus, K., Cohen, C., Chang, M., Lovett, A., \&
Usher, J. (2012). Using CogSketch to support learning cross-sectional
reasoning. Poster presented at the National Science Foundation site
visit for the Spatial Intelligence and Learning Center (SILC).
Philadelphia, PA.

\section{Open Source Software}\label{open-source-software}

\texttt{reservoir\_nn} is a package that enables the use of reservoir
computing architectures in Keras. It enables the flexible creation of
reservoir layers that can be used just like any other type of Keras
layer. \href{https://github.com/keras-team/reservoir_nn}{github}

\texttt{SIMON}: a character-level CNN + LSTM for text classification.
Transfer learn with the model to make inferences about class membership
of text data, e.g., age prediction, spam classification, text similarity
for arbitrary classes, etc.
\href{https://arxiv.org/abs/1901.08456}{arXiv}

CatLearn DIVA: the DIVergent Autoencoder implemented in \texttt{R}
(2016). Available as a module in the \texttt{catlearn} \texttt{R}
Package for computational modelling of formal psychological theories.
\texttt{catlearn} is a framework and archive for distributed
collaboration in formal modeling in psychology.
\href{http://catlearn.r-forge.r-project.org}{r-forge}

Wills, A. J., Edmunds, C. E., Kurtz, K. J., \& Honke, G. A Practical
Introduction to Distributed Collaboration for Formal Modeling: A
Half-day Tutorial. Tutorial at the 50\textsuperscript{th} Annual Meeting
of the Society for Mathematical Psychology, University of Warwick, UK.

Catlearn Supplementals. \texttt{catlearn.suppls} is an \texttt{R}
package that provides a suite of helper functions for cognitive modeling
under the \texttt{catlearn} framework.
\href{http://www.github.com/ghonk/catlearn.suppls}{github}

\section{Teaching}\label{teaching}

\begin{longtable}[]{@{}lll@{}}
\toprule
Course & Role & Semester\tabularnewline
\midrule
\endhead
Research Methods & Discussion Instructor & Fall 2017\tabularnewline
Statistical Analysis and Design & Instructor & Summer
2017\tabularnewline
Experiment Psychology: Perception & Teaching Assistant & Spring
2017\tabularnewline
Cognition Lab & Instructor & Fall 2016\tabularnewline
Experimental Psychology: Cognition & Instructor & Summer
2016\tabularnewline
General Psychology & Teaching Assistant & Spring 2016\tabularnewline
Perception Lab & Instructor & Fall 2015\tabularnewline
Experimental Psychology: Cognition & Teaching Assistant & Fall
2012\tabularnewline
\bottomrule
\end{longtable}

\section{Ad Hoc Reviewing}\label{ad-hoc-reviewing}

Psychophysiology

PLOS One

Acta Psychologica

ICML

Behavorial Research Methods

Cognitive Science Society

Cognitive Processing

Cognitive Psychology

Journal of Experimental Psychology: Learning, Memory, and Cognition

Memory and Cognition

Psychological Science

\end{document}